\documentclass[a4paper]{article}
\usepackage{ucs}
\usepackage[utf8x]{inputenc}
\usepackage{enumitem}
\usepackage{a4wide}
\usepackage{geometry}
\usepackage[automark]{scrpage2}
\pagestyle{scrheadings}
\usepackage{amssymb}
\usepackage{ulem}
\usepackage{graphicx} 
\usepackage{amsmath, amssymb, amstext, amsfonts, mathrsfs}
\usepackage[procnames]{listings}
\usepackage{color}
\usepackage{titlesec}
\usepackage{polynom}
\titleformat*{\section}{\large\bfseries}
\clearscrheadfoot
\usepackage{wasysym} %Fuer den Blitz
\ohead{Mathematik III - Blatt 6 -  NAME}
\cfoot{\pagemark}
\geometry{a4paper,left=10mm,right=10mm, top=2cm, bottom=2cm} 
\parindent0pt
\author{NAME}
\title{\vspace{-2cm}Mathematik III - Blatt 6}
\date{\today{}}

\begin{document}
        
\maketitle

\section*{Aufgabe 1 - 3 Punkte}

 Bestimmen Sie Kern und Bild (jeweils durch Angabe einer Basis) der linearen Abbildung $\alpha: \mathbb{R}^3 \rightarrow \mathbb{R}^3$, die durch folgende Angaben definiert ist:\\
 $\alpha(e_1) = \begin{pmatrix}  -1 \\ 2 \\ 2  \end{pmatrix}, 
 \alpha(e_2) = \begin{pmatrix}  3\\ 0 \\ 6  \end{pmatrix}, 
 \alpha(e_3) = \begin{pmatrix}  5 \\ -4 \\ 2  \end{pmatrix}$\\
 

\section*{Aufgabe 2 - 3 Punkte}
Gegeben sei die Abbildung $\varphi: \{f:\mathbb{R} \rightarrow \mathbb{R}:f \text{ differenzierbar} \} \rightarrow \text{Abb}(\mathbb{R}, \mathbb{R})$, $f \mapsto f'$(Ableitung).
  \begin{enumerate}[label=(\alph*)]
\item Zeigen Sie, dass es sich bei $ \{f:\mathbb{R} \rightarrow \mathbb{R}:f \text{ differenzierbar} \}$ um einen Unterraum von Abb($\mathbb{R},\mathbb{R}$) handelt.
    \item Zeigen Sie, dass Abbildung $\varphi$ linear ist.
    \item Bestimmen Sie den Kern von $\varphi$
 \end{enumerate}


\section*{Aufgabe 3 - 7 Punkte}
Seien $U_1, U_2$ Untervektorräume des Vektorraums $V$. Dann ist $V$ die direkte Summe von $U_1$ und $U_2$ genau dann, wenn $U_1 + U_2 = V$ und $U_1 \cap U_2 = \{0\}$ gilt. 
Schreibe $V = U_1 \bigoplus U_2$
  \begin{enumerate}[label=(\alph*)]
\item Zeigen Sie, dass dim($U_1 \bigoplus U_2) =$dimt($U_1$) + dim($U_2$) gilt und dass sich jeder Vektor aus $U_1 \bigoplus U_2$ eindeutig als Summe eines Vektors aus $U_1$ und eines Vektors aus $U_2$ darstellen lässt.
\item Die Projektion: $\pi U_1 \bigoplus U_2 \rightarrow U_1$ ist definiert durch $u_1 + u_2 \rightarrow u_1$, für alle $u_1 \in U_1, u_2 \in U_2$. Zeigen Sie, dass $\pi$ eine lineare Abbildung ist und bestimmen Sie ihren Rang und ihren Kern.
\item Stellen Sie die Wirkung der Projektion $\pi$ bzgl. der Unterräume $U_1 = \left\langle e_1 \right\rangle, U_2 = \left\langle e_2 \right\rangle \subseteq \mathbb{R}^2$ graphisch dar. ($e_1,e_2$ kanonische Basis von $\mathbb{R}^2$)
\item Seien $U_1 = \left\langle (0,2,3)^t, (1,2,4)^t \right\rangle_\mathbb{R}$ und $U_2 = \left\langle (1,1,1)^t \right\rangle_\mathbb{R}$. Zeigen Sie, dass $\mathbb{R}^3 = U_1 \bigoplus U_2$ gilt. Sei $\pi: U_1 \bigoplus U_2 \rightarrow U_1$ definiert wie in (b). Bestimmen Sie $\pi ((1,0,0)^t)$.
\item Finden Sie zu $U_1$ in (c) einen Untervektorraum $U'_2 \neq U_2$ mit $\mathbb{R}^3 = U_1 \bigoplus U'_2$ und bestimmen Sie $\pi' ((1,0,0)^t)$ für die Projektion $\pi'$ bzgl. dieser direkten Summe.
 \end{enumerate}


\section*{Aufgabe 4 - 4 Punkte}
Sei $\varphi: \mathbb{R}^2 \rightarrow \mathbb{R}^2$ die Abbildung, die die Vektoren des $\mathbb{R}^2$ zuerst um 90° gegen den Uhrzeigersinn dreht, dann um den Faktor 2 streckt und dann and er $x$-Achse spiegelt.

  \begin{enumerate}[label=(\alph*)]
  \item Bestimmen Sie die Bilder der kanonischen Basisvektoren des $\mathbb{R}^2$

  \item Finden Sie eine Matrix $A$ mit $\varphi(v) = A \cdot v$ für alle $v \in \mathbb{R}^2$
  
  \item Entscheiden Sie, ob $\varphi$ bijektiv ist.
  
  \end{enumerate}
 
 \section*{Aufgabe 5 - 3 Punkte}
Sei $A = \begin{pmatrix}
3 & 0&1&2\\
0&9&3&2\\
3&9&2&0 \\
6&3&3&8
\end{pmatrix}$\\
Bestimmen Sie $A$ für 
  \begin{enumerate}[label=(\alph*)]
\item $K = \mathbb{Z}_2$

\item $K = \mathbb{Z}_5$

\item $K = \mathbb{Q}$  
  
 \end{enumerate}

\end{document}