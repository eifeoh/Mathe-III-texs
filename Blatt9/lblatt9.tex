\section*{Aufgabe 1 - 4 Punkte}
  \begin{enumerate}[label=\alph*)]
    \item Geben Sie eine Rekursionsformel für die Anzahl $M(n)$ der Multiplikationen in $K$ an, die bei der Berechnung der Determinante einer $n \times n$ Matrix nach dem Entwicklungssatz von Laplace durchzuführen sind (Multiplikationen mit $(-1)^{i+j}$ werden nicht gezählt).
    \item Können sie eine geschlossene Formel (also eine solche, die nur von $n$, nicht aber von $M(n-1), M(n-2),...,M(1)$ abhängt) angeben und ihre Gültigkeit beweisen? (Das ist keine Frage, sondern ein Befehl...)\\


  \end{enumerate}
\section*{Aufgabe 2 - 6 Punkte}
Gegeben sei die lineare Abbildung $\alpha: \mathbb{R}^3 \rightarrow \mathbb{R}^3, \begin{pmatrix} x\\y\\ z\end{pmatrix} \mapsto \begin{pmatrix} -x + 2y + 2z\\ x-z\\ x-y \end{pmatrix}$
  \begin{enumerate}[label=\alph*)]
    \item Untersuchen Sie $\alpha$ auf Eigenwerte und geben Sie die dazugehörigen Eigenräume an.
    \item  Ist $\alpha$ diagonalisierbar? Wenn ja, geben Sie eine Matrix $S$ an, so dass $S^{-1}A_{\alpha}^{\mathcal{B}}S $Diagonalgestalt hat ($\mathcal{B}$ die kanonische Basis).
  \end{enumerate}

\section*{Aufgabe 3 - 6 Punkte}
Sei $\alpha : \mathbb{R}^3 \rightarrow \mathbb{R}^3$ gegeben durch $(x,y,z)^t \mapsto (-x- \frac{5}{2}y - \frac{1}{2}z, -2x - \frac{1}{2}y + \frac{1}{2}z, 2x + \frac{5}{2}y + \frac{3}{2}z)^t$.
  \begin{enumerate}[label=\alph*)]
    \item Bestimmen Sie die Eigenwerte und Eigenräume von $\alpha$.
    \item Bestimmen Sie alle Fixpunkte von $\alpha$, d.h. alle $v \in \mathbb{R}^3$ mit $\alpha(v) = v$.
  \end{enumerate}
\section*{Aufgabe 4 - 4 Punkte}
Konstruieren sie lineare Abbildungen $\alpha: \mathbb{R}^3 \rightarrow \mathbb{R}^3$ mit den folgenden Eigenschaften.
Geben Sie jeweils die Abbildungsmatrix bezüglich der kanonischen Basis des $\mathbb{R}^3$an, oder zeigen Sie, dass es keine solche lineare Abbildung gibt.
  \begin{enumerate}[label=\alph*)]
    \item $\alpha$ hat die Eigenwerte (EW) 1, 2 und 3.
    \item $\beta$ hat Eigenwert 1 mit Vielfachheit (Vfh) 2 und sonst keine weiteren EWs.
    \item $\gamma$ hat genau einen EW mit Vfh 3 und ist nicht diagonalisierbar.
  \end{enumerate}
\section*{Aufgabe 5 - 3 Punkte}
 Kreuzen Sie bitte an. Für diese Aufgabe sind keine Begründungen erforderlich.
   \begin{enumerate}[label=\alph*)]
    \item In $\mathbb{Z}_{10}$ gilt: $5 \bigodot a = 5 \bigodot b \Rightarrow a = b$
    \item Die Gruppe $\mathbb{Z}_{13}^{*}$ der in ($\mathbb{Z}_{13}, \bigodot$) invertierbaren Elemente enthält genau 12 Elemente.
    \item Sei $K$ ein Körper. Ein Polynom $f \in K[x]$ vom Grad $n$ hat genau $n$ Nullstellen in $K$.
  \end{enumerate}
\section*{Aufgabe 6 - 5 Punkte}
Seien $f = x^3 + 3x^2 , g = \frac{1}{2}x^2 - x \in \mathbb{Q} [x]$. Bestimmen Sie $f+g, f\cdot g, f div g, f \mod g$ und ggT($f,g$).
\section*{Aufgabe 7 - 5 Punkte}
  \begin{enumerate}[label=\alph*)]
    \item Seien $\mathcal{B} = \left \lbrace
      \begin{pmatrix} 5\\0\\2\end{pmatrix},
      \begin{pmatrix} 1\\0\\1\end{pmatrix},
      \begin{pmatrix} -1\\3\\4\end{pmatrix},
      \begin{pmatrix} 0\\7\\-3\end{pmatrix}
      \right \rbrace$ und $\mathcal{C}= \left \lbrace
      \begin{pmatrix} 5\\0\\2\end{pmatrix},
      \begin{pmatrix} 4\\4\\4\end{pmatrix},
      \begin{pmatrix} 6\\-4\\0\end{pmatrix}
      \right \rbrace \subseteq \mathbb{R}^3$. Zeigen Sie, dass weder $\mathcal{B}$ noch $\mathcal{C}$ eine Basis des $\mathbb{R}^3$ ist.
          \item Geben Sie zwei verschiedene Basen  $\tilde{ \mathcal{B}}$ und  $\tilde{ \mathcal{C}}$ des $\mathbb{R}^3$ an, die beide den Vektor $(5,0,2)^t$ enthalten.
          \item Bestimmen Sie die Koordinaten des Vektors $(3,2,1)^t$ bezüglich
       $\tilde{ \mathcal{B}}$ und  $\tilde{ \mathcal{C}}$
  \end{enumerate}
\section*{Aufgabe 8 - 3 Punkte}
Bestimmen Sie Kern und Bild (jeweils durch Angabe einer Basis) der linearen Abbildung $\alpha : \mathbb{R}^3 \rightarrow \mathbb{R}^3$, mit
 $A_{\alpha}^{\mathcal{B}} = \begin{pmatrix} -1 &3&5\\2&0&-4\\2&6&2\end{pmatrix}$ ($\mathcal{B}$ kanonische Basis von $\mathbb{R}^3$)
\section*{Aufgabe 9 - 4 (Bonus) Punkte}
    $S_{\mathcal{C},\mathcal{B}} = \begin{pmatrix}
    1750 & 750 & 0&0&250&1750\\
    600&300&600&600&0&600\\
    0&0&500&0&0&0\\
    6&2&0&2&0&6\\
    6&3&0&0&1&6\\
    1&0&3&4&0&2
    \end{pmatrix}, A_{\alpha}^{\mathcal{B}} = \begin{pmatrix}
    6g&4g&-5g&-4s&-6p&-p\\
    -4g&-g&2g&2s&4p&0\\
    0&0&g&0&0&0\\
    g&g&-2g&-s&-p&0\\
    6g&3g&-6g&-6s&-5p&0\\
    -5g&-4g&5g&4s&5p&p
    \end{pmatrix},\\
     A_{\gamma}^{\mathcal{C}} = \begin{pmatrix}
    1&0&0&0&0&0\\
    0&1&0&0&0&0\\
    0&0&0&0&0&0\\
    0&0&0&0&0&0\\
    0&0&0&0&1&0\\
    0&0&0&0&0&0\\
    \end{pmatrix},
    A_{\beta}^{\mathcal{C}} = E_6 - A_{\gamma}^{\mathcal{C}} $

    Sei $r = \mathcal{K}_{\mathcal{C}}(\alpha($Butter, Zucker, Mehl, Eier, Vanillezucker, Backpulver$)^t$. Rühren Sie $A_{\gamma,\mathcal{C}}$ schaumig und lassen Sie $A_{\gamma,\mathcal{C}} + A_{\beta,\mathcal{C}} r = r$ eine halbe Stunde im Kühlschrank ruhen. Rollen
    Sie $r$ dünn aus und stechen Sie weihnachtliche Formen aus.
    Schmeckt $r$ nach 8 bis 10 Minuten bei 180° im Backofen (ggf. mit Kuchenglasur und Dekoration)?
    ($g$ bezeichne Gramm, $s$ bezeichne Stück und $p$ bezeichne Päckchen).

\end{document}
