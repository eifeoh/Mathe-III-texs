\documentclass[a4paper]{article}
\usepackage{ucs}
\usepackage[utf8x]{inputenc}
\usepackage{enumitem}
\usepackage{a4wide}
\usepackage{geometry}
\usepackage[automark]{scrpage2}
\pagestyle{scrheadings}
\usepackage{amssymb}
\usepackage{ulem}
\usepackage{graphicx} 
\usepackage{amsmath, amssymb, amstext, amsfonts, mathrsfs}
\usepackage[procnames]{listings}
\usepackage{color}
\usepackage{titlesec}
\usepackage{polynom}
\titleformat*{\section}{\large\bfseries}
\clearscrheadfoot
\usepackage{wasysym} %Fuer den Blitz
\ohead{Mathematik III - Blatt 7 -  NONAME}
\cfoot{\pagemark}
\geometry{a4paper,left=10mm,right=10mm, top=2cm, bottom=2cm} 
\parindent0pt
\author{NAME}
\title{\vspace{-2cm}Mathematik III - Blatt 7}
\date{\today{}}

\begin{document}
        
\maketitle

\section*{Aufgabe 1 - 4 Punkte}
Sei $V$ der von $sin(x),cos(x),x$ und $1$ erzeugt Unterraum des $\mathbb{R}$-Vektorraums Abb($\mathbb{R},\mathbb{R}$) und $\delta: V \rightarrow V, f \mapsto f'$ die lineare Abbildung, die eine Funktion $V$ ihre Ableitung zuordnet.

   \begin{enumerate}[label=(\alph*)]
\item Zeigen Sie, dass das Bild von $\delta$ tatsächlich in $V$ liegt.
    \item Bestimmen Sie die Darstellungsmatrix von $\delta$ bezüglich der Basis ($sin(x),cos(x),x,1$), sowie den Rang von $\delta$.
    \item Bestimmen Sie Bild und Kern von $\delta$ und entscheiden Sie, ob $\delta$ injektiv, surjektiv oder bijektiv ist.
 \end{enumerate}
 

\section*{Aufgabe 2 - 7 Punkte}
Die lineare Abbildung $\alpha: \mathbb{R}^3 \rightarrow \mathbb{R}^3$ sei durch...

  \begin{enumerate}[label=(\alph*)]
\item
    \item 
    \item
\item
    \item 
    \item
 \end{enumerate}


\section*{Aufgabe 3 - 2 Punkte}
Sei $V$ ein Vektorraum und $\alpha: V \rightarrow V$ linear.
  \begin{enumerate}[label=(\alph*)]
\item Angenommen es gibt eine Basis $\mathbb{B}$ von $V$, so dass die Darstellungsmatrix $A^{\mathbb{B}}_{\alpha}$ die Einheitsmatrix ist. Ist $\alpha$ dann die Identität?
\item Wie verhält es sich, wenn für die Basis $\mathbb{C} \neq \mathbb{B}$ die Darstellungsmatrix $A^{\mathbb{B},\mathbb{C}}_{\alpha}$die Einheitsmatrix ist?
 \end{enumerate}


\section*{Aufgabe 4 - 3 Punkte}
Seen $V,W$ $K$-Vektorräume

  \begin{enumerate}[label=(\alph*)]
  \item 
  \item 
  
  \end{enumerate}
 
 \section*{Aufgabe 5 - 4 Punkte}
Sei $\alpha$...
\end{document}