\documentclass[a4paper]{article}
\usepackage{ucs}
\usepackage[utf8x]{inputenc}
\usepackage{enumitem}
\usepackage{a4wide}
\usepackage{geometry}
\usepackage[automark]{scrpage2}
\pagestyle{scrheadings}
\usepackage{amssymb}
\usepackage{ulem}
\usepackage{graphicx}
\usepackage{amsmath, amssymb, amstext, amsfonts, mathrsfs}
\usepackage[procnames]{listings}
\usepackage{color}
\usepackage{titlesec}
\usepackage{tikz}
\usetikzlibrary{shapes.misc}
\usepackage{polynom}
\titleformat*{\section}{\large\bfseries}
\clearscrheadfoot
\usepackage{wasysym} %Fuer den Blitz
\ohead{Mathematik III - Blatt 11 -  }
\cfoot{\pagemark}
\geometry{a4paper,left=10mm,right=10mm, top=2cm, bottom=2cm}
\parindent0pt
\author{ }
\title{\vspace{-2cm}Mathematik III - Blatt 11}
\date{\today{}}

\begin{document}

\maketitle

%1
\begin{enumerate}

\item Sei $\mathcal{B}$ eine ONB des $\mathbb{R}^2$ und sei $\alpha: \mathbb{R}^2 \rightarrow \mathbb{R}^2$ eine lineare Abbildung mit 

\begin{center}
$A = A^{\mathcal{B}}_{\alpha} = \frac{1}{\sqrt{2}} \begin{pmatrix} 1& -1 \\ -1 &-1\end{pmatrix}$
\end{center}

\begin{enumerate}[label=(\alph*)]
        \item Zeigen Sie, dass $\alpha$ eine orthogonale Abbildung ist.
        \item Entscheiden Sie, ob es sich um eine Drehung oder eine Spiegelung handelt und geben Sie im Falle einer Drehung den Drehwikel, im Falle einer Spiegelung die Spiegelungsachse an.
\end{enumerate}
%2
\item \begin{enumerate}[label=(\alph*)]
			\item Seien $\alpha,\beta : \mathbb{R}^2 \rightarrow \mathbb{R}^2$ Achsenspiegelungen an zwei Ursprungsgeraden. Zeigen Sie, dass $\alpha \circ \beta$ eine Drehung ist.
			\item Sei $\gamma: \mathbb{R}^2 \rightarrow \mathbb{R}^2$ eine Drehung um den Nullpunkt und $\delta: \mathbb{R}^2 \rightarrow \mathbb{R}^2$ eine Achsenspiegelung an einer Ursprungsgeraden. Zeigen Sie, dass $\gamma \circ \delta$ eine Achsenspiegelung ist. Bestimmen Sie die Spiegelungsachse.

		\end{enumerate}


\item Sei die Abbildung $\alpha : \mathbb{R}^3 \rightarrow \mathbb{R}^3$ gegeben durch $A^{\mathcal{B}}_{\alpha} = 
\begin{pmatrix}
\frac{1}{2} &- \frac{\sqrt{6}}{4} &\frac{\sqrt{6}}{4} \\
\frac{\sqrt{6}}{4} & \frac{3}{4} & \frac{1}{4} \\
- \frac{\sqrt{6}}{4} & \frac{1}{4} & \frac{3}{4}
\end{pmatrix}$
 bezüglich der kanonischen Basis $\mathcal{B}$.
\begin{enumerate}[label=(\alph*)]
\item Zeigen Sie, dass es sich bei der Abbildung $\alpha$ um eine Achsendrehung handelt. 
\item Bestimmen Sie die Drehachse von $\alpha$
\item Bestimmen Sie eine ONB $\mathcal{C}$, sodass $A^{\mathcal{C}}_{\alpha}$ die Form $\begin{pmatrix}
cos(\varphi) & - sin(\varphi) & 0\\
sin(\varphi) & cos(\varphi) & 0\\
0&0&1
\end{pmatrix}$ hat und geben Sie den Winkel $\varphi$ an.
\end{enumerate}

%4
\item \begin{enumerate}[label=(\alph*)]
			\item Sei $\alpha :\mathbb{R}^2 \rightarrow \mathbb{R}^2$ eine affine Abbildung, $\alpha \neq id$. Zeigen Sie, dass $\alpha$ entweder genau einen Fixpunkt, eine Gerade aus Fixpunkten oder keine Fixpunkte hat.
			
			\item Seien $\lambda, \mu \in \mathbb{R}$ und sei $\alpha :\mathbb{R}^2 \rightarrow \mathbb{R}^2$ die affine Abbildung mit 
			\begin{center}
			$\begin{pmatrix} 1 \\ 0 \end{pmatrix} \mapsto \begin{pmatrix} \lambda -2 \\ 2 \end{pmatrix}, \begin{pmatrix} 0 \\ 1 \end{pmatrix} \mapsto \begin{pmatrix} 0 \\ \mu \end{pmatrix} , \begin{pmatrix} 1 \\ 1 \end{pmatrix} \mapsto \begin{pmatrix} \lambda + 1 \\ \mu - 1 \end{pmatrix} $.
			\end{center}
				\begin{enumerate}
				\item Bestimmen Sie $A \in \mathcal{M}_2(\mathbb{R})$ mit $\alpha(v) = A \cdot v +b$ für alle $v \in \mathbb{R}^2$.
				\item Für welche $\lambda, \mu$ hat $\alpha$ genau einen Fixpunkt, eine gerade aus Fixpunkten, bzw. keine Fixpunkte?
				\end{enumerate}

		\end{enumerate}

%5
\item Es ist $y=3x-2$ die Gleichung einer Geraden im $\mathbb{R}^2$.
	\begin{enumerate}[label=(\alph*)]
		\item Geben Sie diese Gerade als affinen Unterraum der Form $w + \left\langle u \right \rangle$ an mit $w,u \in \mathbb{R}^2$.
		\item Sei $\sigma$ die Spiegelung an dieser Geraden. Geben Sie eine orthogonale 2x2-Matrix $A$ und ein $b \in \mathbb{R}^2$ an, so dass $\sigma(v) = A \cdot v + b$ für alle $v \in \mathbb{R}^2$.
	\end{enumerate}

\end{enumerate}
\end{document}